% Created 2017-01-30 Mon 14:41
% Intended LaTeX compiler: pdflatex
\documentclass[presentation,smaller]{beamer}
\RequirePackage{etex}
\RequirePackage[l2tabu,orthodox]{nag}            %% Warn about obsolete commands and packages
\RequirePackage{amsmath,amsfonts,amssymb,amsthm} %% Math
\RequirePackage{ifxetex,ifluatex}                %% Detect XeTeX and LuaTeX
\RequirePackage{fixltx2e}                        %% provides \textsubscript
\RequirePackage{xspace}
\RequirePackage{graphicx}
\RequirePackage{comment}
\RequirePackage{url}
\RequirePackage{relsize}
\RequirePackage{booktabs}
\RequirePackage{tabularx}
\RequirePackage[normalem]{ulem}
\RequirePackage[all]{xy}
\RequirePackage{etoolbox}

%%%
%%% Code Listings
%%%

\RequirePackage{listings}
\lstdefinelanguage{Sage}[]{Python}{morekeywords={True,False,sage,cdef,cpdef,ctypedef,self},sensitive=true}

\lstset{frame=none,
  showtabs=False,
  showspaces=False,
  showstringspaces=False,
  commentstyle={\color{gray}},
  keywordstyle={\color{mLightBrown}\textbf},
  stringstyle ={\color{mDarkBrown}},
  frame=single,
  basicstyle=\tt\scriptsize\relax,
  backgroundcolor=\color{gray!190!black},
  inputencoding=utf8,
  literate={…}{{\ldots}}1,
  belowskip=0.0em,
}

\makeatletter
\patchcmd{\@verbatim}
  {\verbatim@font}
  {\verbatim@font\scriptsize}
  {}{}
\makeatother

%%%
%%% Tikz
%%%

\RequirePackage{tikz,pgfplots}

\usetikzlibrary{calc}
\usetikzlibrary{arrows}
\usetikzlibrary{automata}
\usetikzlibrary{positioning}
\usetikzlibrary{decorations.pathmorphing}
\usetikzlibrary{backgrounds}
\usetikzlibrary{fit,}
\usetikzlibrary{shapes.symbols}
\usetikzlibrary{chains}
\usetikzlibrary{shapes.geometric}
\usetikzlibrary{shapes.arrows}
\usetikzlibrary{graphs}

%% Cache

\ifdefined\tikzcaching  % chktex 1
  \usetikzlibrary{external}
  \tikzexternalize[prefix=build/]
  \tikzset{external/up to date check=diff}  %% MD5 fails from within emacs
\fi

%%%
%%% SVG (Inkscape)
%%%

\ifxetex % chktex 1
\newcommand{\executeiffilenewer}[3]{%
  {\immediate\write18{#3}} % hack
}
\else
\newcommand{\executeiffilenewer}[3]{%
  \ifnum\pdfstrcmp{\pdffilemoddate{#1}}%
    {\pdffilemoddate{#2}}>0%
    {\immediate\write18{#3}}
  \fi%
}
\fi

\newcommand{\includesvg}[2][1.0\textwidth]{%
 \executeiffilenewer{#1.svg}{#1.pdf}%
 {inkscape -z -D --file=#2.svg --export-pdf=#2.pdf --export-latex --export-area-page}%
 \def\svgwidth{#1} 
 \input{#2.pdf_tex}%
} 

%%%
%%% Metropolis Theme
%%%

\usetheme{metropolis}
\metroset{color/block=fill}
\metroset{numbering=none}
\metroset{outer/progressbar=foot}
\metroset{titleformat=smallcaps}

\setbeamercolor{description item}{fg=mLightBrown}
% \setbeamerfont{alerted text}{series=\bfseries}
\setbeamerfont{footnote}{size=\scriptsize}
\setbeamercolor{example text}{fg=mDarkBrown}

\renewcommand*{\UrlFont}{\ttfamily\smaller\relax}

%%%
%%% UTF-8
%%% 

\RequirePackage{unicodesymbols} % after metropolis which loads fontspec

%%%
%%% BibLaTeX
%%%

\RequirePackage[backend=bibtex,
            style=alphabetic,
            maxnames=4,
            citestyle=alphabetic]{biblatex}

\bibliography{local.bib,abbrev3.bib,crypto_crossref.bib,rfc.bib,jacm.bib}

\DeclareFieldFormat{title}{\alert{#1}}
\DeclareFieldFormat[book]{title}{\alert{#1}}
\DeclareFieldFormat[thesis]{title}{\alert{#1}}
\DeclareFieldFormat[inproceedings]{title}{\alert{#1}}
\DeclareFieldFormat[incollection]{title}{\alert{#1}}
\DeclareFieldFormat[article]{title}{\alert{#1}}
\DeclareFieldFormat[misc]{title}{\alert{#1}}

%%% 
%%% Microtype
%%%

\IfFileExists{upquote.sty}{\RequirePackage{upquote}}{}
\IfFileExists{microtype.sty}{\RequirePackage{microtype}}{}

\setlength{\parindent}{0pt}                   %%
\setlength{\parskip}{6pt plus 2pt minus 1pt}  %%
\setlength{\emergencystretch}{3em}            %% prevent overfull lines
\setcounter{secnumdepth}{0}                   %%

%%% Local Variables:
%%% mode: latex
%%% End:
\usepackage{graphicx}
\usepackage{grffile}
\usepackage{longtable}
\usepackage{wrapfig}
\usepackage{rotating}
\usepackage[normalem]{ulem}
\usepackage{amsmath}
\usepackage{textcomp}
\usepackage{amssymb}
\usepackage{capt-of}
\usepackage{hyperref}
\usepackage{microtype}
\usepackage{newunicodechar}
\usepackage{unicodesymbols}
\usepackage[notions,operators,sets,keys,ff,adversary,primitives,complexity,asymptotics,lambda,landau,advantage]{cryptocode}
\usepackage{xspace}
\usepackage{units}
\usepackage{nicefrac}
\usepackage{gensymb}
\usepackage{amsthm}
\usepackage{amsmath}
\usepackage{amssymb}
\usepackage{xcolor}
\usepackage{listings}
\usepackage[color=yellow!40]{todonotes}
\renewcommand{\vec}[1]{\mathbf{#1}\xspace}
\newcommand{\mat}[1]{\mathbf{#1}\xspace}
\DeclareMathOperator{\Vol}{Vol}
\usetheme{default}
\author{Martin R. Albrecht}
\date{Oxford Lattice School}
\title{Attacks on LWE}
\hypersetup{
pdfauthor={Martin R. Albrecht},
pdftitle={Attacks on LWE},
pdfkeywords={},
pdfsubject={},
pdfcreator={Emacs 25.1.1 (Org mode 9.0.4)},
pdflang={English},
colorlinks,
citecolor=gray,
filecolor=gray,
linkcolor=gray,
urlcolor=gray
}
\begin{document}

\maketitle
\begin{frame}{Outline}
\tableofcontents
\end{frame}



\section{Lattice Point Enumeration}
\label{sec:orga0dda1f}

\begin{frame}[label={sec:orgf33c3bd}]{Finding Shortest Vectors}
Given some lattice \(Λ(\mat{B})\), find \(\vec{v} \in Λ(\mat{B})\) with \(\vec{v} \neq 0\) such that \(\|\vec{v}\|^2\) is minimal.
\end{frame}

\begin{frame}[label={sec:org65709b5}]{Finding Short Vectors}
Given some \textbf{matrix} \(\mat{B}\) and some \textbf{bound} \(R\), find \(\vec{v} = \sum_{i=1}^{d} v_i \vec{b}_i\) where at least one \(v_i \neq 0\) such that \(\|\vec{v}\|^2 \leq R^2\).
\end{frame}

\begin{frame}[fragile,label={sec:org0a6eac9}]{Rephrasing in Gram-Schmidt Basis}
 \begin{columns}[t]
\begin{column}{0.6\columnwidth}
Given some basis \(\mat{B}\) for some lattice \(Λ(\mat{B})\) we can compute the Gram-Schmidt orthogonalisation \[\mat{B} = μ \cdot \mat{B}^*\]

Any vector in \(\vec{w} \in Λ(B)\) can be written as 
\begin{align*}
\vec{w} &= \sum_{i=1}^d v_i \vec{b}_i = \sum_{i=1}^{d} v_i \left(\vec{b}_i^* + \sum_{j=1}^{i-1} \mu_{ij} \vec{b}_j^* \right)\\
        &= \sum_{j=1}^{d} \left(v_j  + \sum_{i=j+1}^{d} v_i\, \mu_{ij} \right) \vec{b}_j^* 
\end{align*}
\end{column}

\begin{column}{0.4\columnwidth}
\lstset{language=sage,label= ,caption= ,captionpos=b,numbers=none}
\begin{lstlisting}
B = matrix(ZZ, [[-1,  1, -2], 
                [ 0, -2,  0], 
                [10, -1, -2]])
Bs, mu = B.gram_schmidt()
Bs
\end{lstlisting}

\begin{verbatim}
[   -1     1    -2]
[ -1/3  -5/3  -2/3]
[ 44/5     0 -22/5]
\end{verbatim}


\lstset{language=sage,label= ,caption= ,captionpos=b,numbers=none}
\begin{lstlisting}
v = vector([1,2,3])
v*B == v*(mu*Bs) == (v*mu)*Bs
\end{lstlisting}

\begin{verbatim}
True
\end{verbatim}
\end{column}
\end{columns}
\end{frame}

\begin{frame}[fragile,label={sec:org66fd598}]{Orthogonal Projections}
 \begin{columns}[t]
\begin{column}{0.55\columnwidth}
The same representation applies to projections of \(\vec{w}\):

\begin{align*}
\pi_k\left(\vec{w}\right) &= \pi_k\left(\sum_{i=1}^{d} v_i \left(\vec{b}_i^* + \sum_{j=1}^{i-1} \mu_{ij} \vec{b}_j^* \right)\right)\\
                        &= \sum_{j=\alert{k}}^{d} \left(v_j  + \sum_{i=j+1}^{d} v_i\, \mu_{ij} \right) \vec{b}_j^*
\end{align*}
\end{column}

\begin{column}{0.45\columnwidth}
\lstset{language=sage,label= ,caption= ,captionpos=b,numbers=none}
\begin{lstlisting}
k, d = 1, 3
w_1 = 0
for j in range(k, d):
    c = v[j]
    for i in range(j+1, d):
        c += v[i]*mu[i,j]
    w_1 += c*Bs[j]
w_1
\end{lstlisting}

\begin{verbatim}
(155/6, -17/6, -43/3)
\end{verbatim}

\lstset{language=sage,label= ,caption= ,captionpos=b,numbers=none}
\begin{lstlisting}
def proj(u, v):
    return v*u/(u*u) * u

w = v * mu * Bs
w - proj(Bs[0], w)
\end{lstlisting}

\begin{verbatim}
(155/6, -17/6, -43/3)
\end{verbatim}
\end{column}
\end{columns}
\end{frame}

\begin{frame}[fragile,label={sec:orgee37204}]{Bounding Norms}
 \begin{columns}[t]
\begin{column}{0.6\columnwidth}
Since \(\vec{b}_i^*\) are orthogonal, we can write:

\begin{align*}
\|π_k\left(\vec{w}\right)\|^2 &= \left\|\sum_{j=k}^{d} \left(v_j  + \sum_{i=j+1}^{d} v_i\, \mu_{ij} \right) \vec{b}_j^*\right\|^2\\
&= \sum_{j=k}^{d} \left(v_j  + \sum_{i=j+1}^{d} v_i\, \mu_{ij} \right)^2 \|\vec{b}_j^*\|^2
\end{align*}



Thus \[\|π_{k}(\vec{w})\| ≥ \|π_{k+1}(\vec{w})\|,\] i.e. vectors don’t become longer by projecting.
\end{column}

\begin{column}{0.4\columnwidth}
\lstset{language=sage,label= ,caption= ,captionpos=b,numbers=none}
\begin{lstlisting}
k, d = 1, 3
r = 0
for j in range(k, d):
    c = v[j]
    for i in range(j+1, d):
        c += v[i]*mu[i,j]
    r += c^2 * abs(Bs[j])^2
r
\end{lstlisting}

\begin{verbatim}
5285/6
\end{verbatim}

\lstset{language=sage,label= ,caption= ,captionpos=b,numbers=none}
\begin{lstlisting}
def proj(u, v):
    return v*u/(u*u) * u

w = v * mu * Bs
abs(w - proj(Bs[0], w))^2
\end{lstlisting}

\begin{verbatim}
5285/6
\end{verbatim}
\end{column}
\end{columns}
\end{frame}

\begin{frame}[label={sec:orgbee23f9}]{Key Idea}
From \[\|π_{d}(\vec{w})\|^2 \leq \|π_{d-1}(\vec{w})\|^2 ≤ … ≤ \|π_{1}(\vec{w})\|^2 ≤ \|\vec{w}\|^2 \leq R^2,\] find candidates for \(π_{k+1}(\vec{w})\) and extend solution to \(π_{k}(\vec{w})\) using
\begin{align*}
\pi_k\left(\vec{w}\right) &= \sum_{j=k}^{d} \left(v_j  + \sum_{i=j+1}^{d} v_i\, \mu_{ij} \right) \vec{b}_j^*\\
&=  \pi_{k+1}(\vec{w}) + \left(v_k  + \sum_{i=k+1}^{d} v_i\, \mu_{ik} \right) \vec{b}_k^*
\end{align*}
and
\begin{align*}
\|\pi_k\left(\vec{w}\right)\|^2 
&=  \|\pi_{k+1}(\vec{w})\|^2 + \left(v_k  + \sum_{i=k+1}^{d} v_i\, \mu_{ik} \right)^2 \|\vec{b}_k^*\|^2
\end{align*}
\end{frame}

\begin{frame}[fragile,label={sec:org90feb2a}]{Execution}
 \begin{columns}[t]
\begin{column}{0.58\columnwidth}
From the bound \(R\) we know \[v_d^2 \|\vec{b}_d^*\|^2 = \|π_d(\vec{w})\|^2 ≤ R^2\]

Thus, the only valid candidates for \(v_d\) are \[\ZZ \cap [-R/\|\vec{b}_d^*\|,R/\|\vec{b}_d^*\|]\]

For any choice of \(v_d\) in this interval, we know
\begin{align*}
\|π_{d-1}(\vec{w})\|^2 \leq& R^2\\
v_d^2 \|\vec{b}_d^*\|^2 + (\alert{v_{d-1}} + v_d\, \mu_{d,d-1})^2 \cdot \|\vec{b}_{d-1}^*\|^2 \leq& R^2\\ 
\end{align*}

This defines an integral interval for \(v_{d-1}\)
\end{column}

\begin{column}{0.42\columnwidth}
\lstset{language=sage,label= ,caption= ,captionpos=b,numbers=none}
\begin{lstlisting}
R = abs(B[0])
bnd = floor(abs(Bs[-1])/R)
range(-bnd, bnd+1)
\end{lstlisting}

\begin{verbatim}
[-4, -3, -2, -1, 0, 1, 2, 3, 4]
\end{verbatim}

\lstset{language=sage,label= ,caption= ,captionpos=b,numbers=none}
\begin{lstlisting}
v_d = 0
c = -v_d*mu[-1,-2]
o = R^2 - v_d^2*abs(Bs[-1])^2
o = sqrt(o)/abs(Bs[-2])
range(ceil(c-o), floor(c+o)+1)
\end{lstlisting}

\begin{verbatim}
[-1, 0, 1]
\end{verbatim}

…
\end{column}
\end{columns}
\end{frame}

\begin{frame}[label={sec:org1a9b7d0}]{Enumeration}
\begin{description}
\item[{shortest vectors}] reduce \(R\) whenever a shortet vector than bound is found
\item[{short enough vectors}] stop when vector with target norm is found
\item[{target radius}] \(R = \|\vec{b}_1\|\) always works, picking a small \(R\) reduces the search space, e.g. \(R ≈ \Vol(L)^{1/d}\)
\item[{pruning}] not all choices for \(v_k\) equally likely lead to a solution, skip some
\end{description}
\end{frame}


\section{BKZ}
\label{sec:org7879d80}

\section{LWE}
\label{sec:org583f538}

\begin{frame}[label={sec:org858de6a}]{Learning with Errors}
Let \(n,\,q\) be positive integers, \(\chi\) be a probability distribution on \(\ZZ\) and \(\vec{s}\) be a secret vector in \(\ZZ_q^n\). We denote by \(L_{n,q,\chi}\) the probability distribution on \(\ZZ_q^n × \ZZ_q\) obtained by choosing \(\vec{a} ∈ \ZZ_q^n\) uniformly at random, choosing \(e ∈ \ZZ\) according to χ and considering it in \(\ZZ_q\), and returning \((\vec{a}, c) = (\vec{a}, \Angle{\vec{a},\vec{s}}+ e) ∈ \ZZ_q^n × \ZZ_q\).

\begin{description}
\item[{Decision-LWE}] is the problem of deciding whether pairs \((\vec{a}, c) ∈ \ZZ_q^n × \ZZ_q\) are sampled according to \(L_{n, q, \chi}\) or the uniform distribution on \(\ZZ_q^n × \ZZ_q\).

\item[{Search-LWE}] is the problem of recovering \(\vec{s}\) from \((\vec{a}, c)=(\vec{a}, \Angle{\vec{a},\vec{s}} + e) ∈ \ZZ_q^n × \ZZ_q\) sampled according to \(L_{n, q, \chi}\).
\end{description}
\end{frame}

\section{Dual Lattice Attack}
\label{sec:org31e39da}
\begin{frame}[label={sec:orgaae87bf}]{Short Integer Solutions}
Consider the scaled (by \(q\)) dual lattice: \[q Λ^* = \{ \vec{x} \in \mathbb{Z}^m \enspace | \enspace \vec{x} \vec{A} \equiv 0 \bmod q\}.\] A short vector of \(qΛ^*\) is equivalent to solving SIS on \(\vec{A}\).

\begin{block}{Short Integer Solutions (SIS)}
Given \(q \in \mathbb{Z}\), a matrix \(\vec{A}\), and \(t < q\); find \(\vec{y}\) with \(0 < \|\vec{y}\| \leq t\) and \[\vec{y}\, \vec{A} \equiv  \vec{0} \pmod{q}.\]
\end{block}
\end{frame}

\begin{frame}[label={sec:org1937f59}]{Strategy}
\begin{itemize}
\item Find a short \(\vec{y}\) solving SIS on \(\vec{A}\).
\item Given LWE samples \(\vec{A}, \vec{c}\) where either \(\vec{c} = \vec{A}\vec{s} + \vec{e}\) or \(\vec{c}\) uniformly random.
\item Compute \(\Angle{\vec{y}, \vec{c}}\). 
\begin{itemize}
\item If \(\vec{c} = \vec{A} \cdot \vec{s} + \vec{e}\), then \(\Angle{\vec{y}, \vec{c}} = \Angle{\vec{y}\vec{A}, \vec{s}} + \Angle{\vec{y}, \vec{e}} \equiv \Angle{\vec{y}, \vec{e}} \pmod{q}\).
\item If \(\vec{c}\) is uniformly random, so is \(\Angle{\vec{y}, \vec{c}}\).
\end{itemize}
\end{itemize}

If \(\vec{y}\) is sufficiently short, then \(\Angle{\vec{y}, \vec{e}}\) will also be short, since \(\vec{e}\) is also small, and can be distinguished from uniform values.
\end{frame}

\begin{frame}[label={sec:org897ea0b}]{Degrees of Freedom}
\end{frame}
\begin{frame}[label={sec:org3cccd4a}]{Constructing a Basis}
\end{frame}
\begin{frame}[label={sec:orgddc5b70}]{Lattice Reduction}
A \alert{reduced lattice} basis is made of short vectors, in particular the first vector.
\begin{itemize}
\item Construct a basis of the dual from the instance.
\item Feed to a lattice reduction algorithm to obtain short vectors \(\vec{v}_i\).
\item Check if \(\vec{v}_i\, \vec{A}\) are small.
\end{itemize}
\end{frame}

\begin{frame}[label={sec:orge5d246f}]{BKW}
\end{frame}
\section{Primal Lattice Attack (uSVP Version)}
\label{sec:orgac664b3}

\begin{frame}[standout,label={sec:org4463593}]{Fin}
\begin{center}
\Huge \alert{Thank You}
\end{center}
\end{frame}
\end{document}