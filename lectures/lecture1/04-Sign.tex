%!TEX root = main.tex

\section{Signatures are tricky}

\begin{frame}
\frametitle{Solution 1: Hash-Then-Sign}
{\bf Sign}
\begin{itemize}
  \item Hash the message to a random vector $\ppurple{\vec m}$.
  \item apply \textsc{GaussianSampling} with a good basis $\bblue{\vec G}$: \\
    \quad find $\oorange{\vec s} \in L$ close to $\ppurple{\vec m}$ .
\end{itemize}
\vspace{.2cm}
{\bf Verify}
\begin{itemize}
  \item check that $\oorange{\vec s} \in L$ using the bad basis $\rred{\vec B}$
    \item and that $\ppurple{\vec m}$ is close to $\oorange{\vec s}$.
\end{itemize}


\end{frame}


\begin{frame}
\frametitle{Ad-hoc construction of lattices with a good basis}
\begin{definition}[The Matrix-NTRU assumption]
  For two small matrices $\bblue{\vec F}, \bblue{\vec G} \gets \chi^{n \times n}$, set $\vec H = \vec F\vec G^{-1} \bmod q$. \\
  Distinguishing $\vec H$ from uniform is hard.\footnote{$\vec H$ is provably uniform for midly large $\bblue{\vec F}, \bblue{\vec G}$ [Stehle Steinfeld 2012]}
\pause
\begin{alertblock}{Do not overstreched !}
  Can be much weaker than (Ring) LWE for large $q$. \\
   cf. Thursday : [A. Bai D. 2016, Kirchner Fouque 2016]
\end{alertblock}

\end{definition}
\pause

\begin{itemize}
  \item $(\bblue{\vec F}, \bblue{\vec G})$ is a good partial basis of the lattice.
  \item It can be completed into a full good basis. \\ optimal parameters studied in [D. Prest Lyubashevski 2013]\footnote{IMHO: Precise parameter proposal not conservative enough}
\end{itemize}
\end{frame}

\begin{frame}
\frametitle{Provably secure construction of lattices with a good basis}
SoA: [Micciancio Peikert 2012] ``Simpler, Tighter, Faster, Smaller''.
\begin{itemize}
  \item Define a Gadget matrix $\vec G = [\vec I, 2\vec I, 4\vec I, \dots 2^k \vec I]$
  \item Start from a truly random matrix $\vec A$
  \item Extend $\vec A$ to $\vec A' = [\vec A | \vec R \vec A + \vec G]$ for a small matrix $\vec R$
  \item $\vec A'$ is statistically uniform {\hfill \small (leftover hash lemma)}
  \item $\vec R$ provides a good basis of $\Lambda^\bot(\vec A)$
\end{itemize}

\begin{itemize}
\item[+] Many extensions (tags, basis delegation)
\item[+] Very convenient for advanced crypto
\item[--] Cumbersome for basic crypto
\end{itemize}

\end{frame}

\begin{frame}
\frametitle{Good Gaussian Sampling in Practice ?}
\begin{itemize}
  \item[+] Leads to the most compact lattice signature schemes
  \item[+] Good asymptotic complexity \hfill{\scriptsize FFO [D. Prest 2016]}
  \item[--] Requires Floating-Point Arithmetic
\end{itemize}
\vspace{.5cm}
\pause
Not so studied in practice so far \dots \\
{\bf Wide impact:} signatures, homomorphic signatures, IBE, ABE, \dots

\end{frame}

\begin{frame}
\frametitle{Solution 2: Fiat-Shamir transform}

{\bf Idea: } [Lyubashevski, \dots, BLISS, TESLA] 
\begin{itemize}
  \item Prove knowledge of a short vector without revealing it 
\end{itemize}

\begin{itemize}
  \item[+] No need for a full basis
  \item[+] Sampling potentially simpler
  \item[--] Larger signatures.
\end{itemize}
\end{frame}
