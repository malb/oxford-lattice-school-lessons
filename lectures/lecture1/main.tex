\documentclass{beamer} % "Beamer" is a word used in Germany to mean video projector. 
\usepackage{lmodern}
\usetheme{Luebeck} % Search online for beamer themes to find your favorite or use the Berkeley theme as in this file.
\usepackage[utf8]{inputenc}

\usepackage{color} % It may be necessary to set PCTeX or whatever program you are using to output a .pdf instead of a .dvi file in order to see color on your screen.
\usepackage{graphicx} % This package is needed if you wish to include external image files.
\usepackage[absolute,overlay]{textpos}
\usepackage{tikz}
\usetikzlibrary{calc} 
\usepackage{eurosym}



\xdefinecolor{redx}{rgb}{1,0,0}
\xdefinecolor{greenx}{rgb}{0,.7,0}
\xdefinecolor{bluex}{rgb}{0,0,.8}
\xdefinecolor{purplex}{rgb}{.7,0,.7}
\xdefinecolor{orangex}{rgb}{.9,.5,0}

\newcommand{\rgrey}[1]{{\color{greyx!40}#1}}
\xdefinecolor{greyx}{rgb}{0,0,0}

\newcommand{\rred}[1]{{\color{redx!100}#1}}
\newcommand{\ggreen}[1]{{\color{greenx!100}#1}}
\newcommand{\bblue}[1]{{\color{bluex!100}#1}}
\newcommand{\ppurple}[1]{{\color{purplex!100}#1}}
\newcommand{\oorange}[1]{{\color{orangex!100}#1}}
\renewcommand{\vec}{\mathbf}


\setbeamertemplate{headline}{}
\addtobeamertemplate{block begin}{\pgfsetfillopacity{0.88}}{\pgfsetfillopacity{1}}
 \addtobeamertemplate{block alerted begin}{\pgfsetfillopacity{0.9}}{\pgfsetfillopacity{1}}
 \addtobeamertemplate{block example begin}{\pgfsetfillopacity{0.92}}{\pgfsetfillopacity{1}}

\title{Overview of Lattice based Cryptography}
\author[Léo Ducas, CWI, Amsterdam, The Netherlands]{\Large Léo Ducas} 
\institute{CWI, Amsterdam, The Netherlands \\
\includegraphics[height = 1.8cm]{img/CWI.png}
}
\date{Spring School on Lattice-Based Cryptography \\
Oxford, March 2017} 


\AtBeginSection[]{%
  \begin{frame}<beamer>
    \frametitle{Outline}
    \tableofcontents[sectionstyle=show/shaded,subsectionstyle=hide/show/hide]
  \end{frame} 
}

\begin{document} 

\usebackgroundtemplate{\includegraphics[width=1\paperwidth]{img/metamorphosis.png}}
\begin{frame}
\titlepage
\end{frame}
\usebackgroundtemplate{}

\section{The Geometric point of view}

\begin{frame}
  \frametitle{Lattices !}
\begin{figure}
\includegraphics{tikz_lattice.pdf}
\end{figure}
\begin{definition}
  A lattice $L$ is a discrete subgroup of a finite-dimensional Euclidean vector space.
\end{definition}
\end{frame}

\begin{frame}
\frametitle{Bases of a Lattice}
\includegraphics{tikz_goodbadbasis.pdf}
\\
\begin{tabular}{cccll}
 $\bblue{\vec G}$ & $ \rightarrow$ & $\rred{\vec B}$& : easy &(randomization); \\
 $\rred{\vec B} $ & $\rightarrow$ & $\bblue{\vec G}$& : hard &(LLL, BKZ, Lattice Sieve...).
\end{tabular}
\end{frame}

\begin{frame}
\frametitle{Bases and Fundamental Domains}
Each Basis defines a {\bf parallelepipedic tiling}.
% Les algorithmes de Babai découpent l'espace en domaines fondamentaux {\bf parallélépipédiques} selon une base donnée.
\includegraphics{tikz_goodbadtiling.pdf}
\\
{\bf Round'off Algorithm [Lenstra, Babai]}:
\begin{itemize}
  \item<2-> Given a target \ppurple{$\vec t$}
  \item<3-> Find's $\oorange{\vec v} \in L$ at the center the tile.
\end{itemize}
\end{frame}





\begin{frame}
\frametitle{Round'off Algorithm}
\begin{tabular}{ccc}
\only<1-3>{\includegraphics{tikz_ro1.pdf}}
\only<4->{\includegraphics{tikz_ro4.pdf}}

& \raisebox{2cm}{
$\begin{matrix}
\uncover<2->{\times {\vec B}^{-1}} \\
\longrightarrow \\
  \\
  \\
\uncover<4->{\longleftarrow} \\   
\uncover<4->{\times \vec B} \\
\end{matrix}
$}
&
\uncover<2->{
  \only<1-2>{\includegraphics{tikz_ro2.pdf}}
  \only<3->{\includegraphics{tikz_ro3.pdf}}  
 }
\end{tabular}
\textsc{RoundOff} Algorithm [Lenstra,Babai]:
\begin{itemize}
\item<2-> Use $\vec B$ to switch to the lattice $\mathbb Z^n$ ($\times \vec B^{-1}$)
\item<3-> round each coordinate (square tiling)
\item<4-> switch back to $L$ ($\times \vec B$)
\end{itemize}
\[\uncover<2->{\oorange{\vec t'} = \vec B^{-1} \cdot \oorange{\vec t} ;} 
\quad \uncover<3->{\ppurple{\vec v'} = \lfloor \oorange{\vec t'} \rceil;}
\quad \uncover<4->{\ppurple{\vec v} = \vec B \cdot \ppurple{\vec v'}} \] 
\end{frame}


\begin{frame}
\frametitle{Nearest-Plane Algorithm}
There is a better algorithm (\textsc{NearestPlane}) based on Gram-Schmidt Orth. $\vec B^\star$ of a basis $\vec B$:

\begin{figure}
\includegraphics{tikz_goodbadNP.pdf}
\end{figure}
\begin{itemize}
  \item Worst-case distance: $\frac 1 2 \sqrt {\sum \|\vec b_i^\star\|^2}$ \hfill (Approx-CVP)
  \item Correct decoding of $\oorange{\vec t} = \ppurple{\vec v} + \vec e$ where $\vec v \in \Lambda$ if \hfill (BDD) \[\|\vec e\| \leq \min \|\vec b_i^\star\| \] 
\end{itemize}

\end{frame}

 
\begin{frame}
\frametitle{Finding Close Vectors}
With a good basis $\bblue{\vec G}$ one can solve Approx-CVP / BDD.\\
Given only a bad basis $\rred{\vec B}$, solving CVP is a {\bf hard problem}. \vspace{.4cm}\\

\includegraphics{tikz_goodbadCVP.pdf}
\vspace{.4cm}\\
Can this somehow be used as a trapdoor ?
\end{frame}


\begin{frame}
  \frametitle{Encryption from lattices (simplified)}
  Using the (second) decoding algorithm, on can recover $\vec v, \vec e$ from $\vec w = \vec v + \vec e$ when 
 \[ \|\vec e \| \leq \min \| \vec b_i^*\| \]


Fix a parameter $\eta$:
\begin{itemize}
  \item Private key: good basis $\bblue{\vec G}$ such that $\|\vec g_i^*\| \geq \eta$
  \item Public key: bad basis $\rred{\vec B}$ such that $\|\vec b_i^*\| \ll \eta$
  \item Message : $\vec m \in \Lambda = \mathcal L(\rred{\vec B}) = \mathcal L(\bblue{\vec G})$
  \item Ciphertext : $\vec c = \vec m + \vec e$, for a random error $\vec e$, $\|\vec e\| \leq \eta$
  \item Decryption : $(\vec m', \vec e) = \textsc{NearestPlane}(\vec c)$
\end{itemize}

\end{frame}




\begin{frame}
  \frametitle{Encryption from lattices}
  
Decryption : $(\vec m', \vec e) = decode(\vec c)$\\
\includegraphics{tikz_goodbadDEC.pdf}
\begin{itemize}
  \item With the good basis $\bblue{\vec G}$, $\vec m' = \vec m$
  \item With the bad basis $\rred{\vec B}$, $\vec m' \neq \vec m$ : decryption fails !
\end{itemize}

\end{frame}


\begin{frame}
\frametitle{Signatures}
%{\bf Idée:} Schéma ``dual'' du chiffrement basé sur les réseaux. \vspace{.3cm}

{\bf Sign} %($\approx$ déchiffrement)
\begin{itemize}
  \item Hash the message to a random vector $\ppurple{\vec m}$.
  \item apply Round'off with a good basis $\bblue{\vec G}$: \\
    \quad find $\oorange{\vec s} \in L$ close to $\ppurple{\vec m}$ .
\end{itemize}
\vspace{.2cm}
{\bf Vérify}
\begin{itemize}
  \item check that $\oorange{\vec s} \in L$ using the bad basis $\rred{\vec B}$
    \item and that $\ppurple{\vec m}$ is close to $\oorange{\vec s}$.
\end{itemize}
\vspace{.2cm}
\includegraphics{tikz_goodbadSIGN.pdf}
\end{frame}


\begin{frame}
\frametitle{A statistical attack~[NguReg05,DucNgu12]}
The difference $\vec s - \vec m$ is always inside the parallelepiped spanned by the good basis $\bblue{\vec G}$ (or its GSO $\bblue{\vec G^\star}$):
\begin{figure}
{\centering{\includegraphics[width=6cm]{img/learning.jpg}}}
\end{figure}
Each signatures $(\vec s,\vec m)$ leaks a bit of information about the secret key $\bblue{\vec G}$. \\
Nguyen et Regev showed how to ``learn the parralepiped'' using a few signatures: \\
$\quad \Rightarrow$ Total break of original GGH and NTRUSign schemes. \\
\end{frame}


\begin{frame}
\frametitle{Gaussian sampling}
Using the appropriate randomized rounding (Gaussian-sampling) 
the distribution $\vec s - \vec m$ can be made {\bf independent} of $\bblue{\vec G}$
\begin{itemize}
  \item{} [Klein 2000, Gentry Peikert Vaikuthanathan 2008]: \\
  {\scriptsize Slow and memory heavy, even in the {\bf ring-setting} (NTRU, Ring-LWE)}
  \item{} [Peikert 2010] \\
  {\scriptsize Faster and less memory, but worse quality}
  \item{} [DucPre15] (Fast Fourier Orthopgonalization) \\
  {\scriptsize Fast and good quality for certain rings}

\end{itemize}
\end{frame}


\end{document}


